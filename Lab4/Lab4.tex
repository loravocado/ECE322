\documentclass[12pt, letterpaper, titlepage]{article}
\usepackage[utf8]{inputenc}

\usepackage{geometry}
\usepackage{color,graphicx,overpic,colortbl} 
\usepackage{fancyhdr}
\usepackage{amsmath,amsthm,amsfonts,amssymb}
\usepackage{mathtools}
\usepackage{hyperref}
\usepackage{multicol}
\usepackage{array}
\usepackage{float}
\usepackage{blindtext}
\usepackage{longtable}
\usepackage{scrextend}
\usepackage[font=small,labelfont=bf]{caption}
\usepackage{calc}
\usepackage{titlesec}
\usepackage{listings}
\usepackage[normalem]{ulem}
\usepackage{tabularx}
\usepackage{mathrsfs}
\usepackage{bookmark}
\usepackage{setspace}
\usepackage{ragged2e}
\usepackage{ltablex}
\usepackage{xurl}
\usepackage{tikz}
\usepackage{pgfplots}
\usepackage{xparse}

\pgfplotsset{width=8cm,compat=1.15}\usepgfplotslibrary{patchplots}
\mathtoolsset{showonlyrefs}  
\allowdisplaybreaks

\definecolor{mycolor}{rgb}{0, 0, 0}

\geometry{top=2.54cm, left=2.54cm, right=2.54cm, bottom=2.54cm}
\setlength{\headheight}{20pt}
\setlength{\parskip}{0.3cm}
\setlength{\parindent}{1cm}

\pagestyle{fancy}
\fancyhf{}
\rhead{Lora Ma - 1570935}
\lhead{\textit{ECE 322 Lab 3}}
\rfoot{Page \thepage}

\begin{document} 
\singlespacing

\section{Introduction}
The purpose of this lab is to become familiar with integration white-box testing. The goal of this lab was to give us experience with using Python and Python unittest unittest.mock for integration testing.

Integration testing serves as a logical extensions of unit testing. There are two general approaches to integration testing -- non-incremental testing (Big Bang) and incremental testing (Top down/Bottom up). In non-incremental testing, each module is tested individually and then the whole system is tested as a whole. In integration testing, we combine the next module to be tested with the set of previously tested modules before running tests. This can generally done in either a bottom up or top down method. A bottom up method involves testing the lowest level modules in isolation and then incrementally adding higher and higher module levels. A top down method involves testing the highest level modules in isolation and then incrementally adding lower modules.

Integration testing usually involves using various stubs and drivers. Stubs, in integration testing, is used as a stand in for lower level modules that are not currently under test. A stub returns a dummy value or makes an assertion so that the test case can ensure it was called. Drivers are a piece of testing code which makes it possible to call a submodule of an application independently.

In this lab, we will be focusing on non-incremental testing

\section{Non-incremental testing - Big Bang testing}
During the non-incremental testing portion of this lab, we made unit tests for module A-F. The tests can be found in Appendix A. The test results can be found on the following page

\section{Conclusion and Discussion}


\end{document}