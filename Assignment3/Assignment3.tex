\documentclass[12pt, letterpaper, titlepage]{article}
\usepackage[utf8]{inputenc}
\usepackage{geometry}
\usepackage{color,graphicx,overpic} 
\usepackage{fancyhdr}
\usepackage{amsmath,amsthm,amsfonts,amssymb}
\usepackage{mathtools}
\usepackage{hyperref}
\usepackage{multicol}
\usepackage{array}
\usepackage{float}
\usepackage{blindtext}
\usepackage{longtable}
\usepackage{scrextend}
\usepackage[font=small,labelfont=bf]{caption}
\usepackage[framemethod=tikz]{mdframed}
\usepackage{calc}
\usepackage{titlesec}
\usepackage{listings}
\usepackage[normalem]{ulem}
\usepackage{tabularx}
\usepackage{mathrsfs}
\usepackage{bookmark}
\usepackage{setspace}
\usepackage{tabularx}
\usepackage{ltablex}

\mathtoolsset{showonlyrefs}  
\allowdisplaybreaks

\definecolor{mycolor}{rgb}{0, 0, 0}

\geometry{top=2.54cm, left=2.54cm, right=2.54cm, bottom=2.54cm}
\setlength{\headheight}{20pt}
\setlength{\parskip}{0.3cm}
\setlength{\parindent}{1cm}

\pagestyle{fancy}
\fancyhf{}
\rhead{Lora Ma - 1570935}
\lhead{\textit{ECE 322 Assignment 4}}
\rfoot{Page \thepage}

\begin{document} 
\singlespacing

\section{Q1}
We have derived the following equations from the graph:
\begin{align}
    p_\text{A} &= 0.8p_\text{H} \\
    p_\text{B} &= 0.7p_\text{A} + 0.5p_\text{E} \\
    p_\text{C} &= 0.7 p_\text{D} + p_\text{B} \\
    p_\text{D} &= 0.5p_\text{E} \\
    p_\text{E} &= 0.3p_\text{A} \\
    p_\text{F} &= 0.6p_\text{C} \\
    p_\text{G} &= 0.2p_\text{H} + 0.4p_\text{C} + p_\text{F} \\
    p_\text{H} &= p_\text{G} + 0.3p_\text{D} \\ 
    1 &= p_\text{A} + p_\text{B} + p_\text{C} + p_\text{D} + p_\text{E} + p_\text{F} + p_\text{G} + p_\text{H} \\
\end{align}
Solving these equations gets us the following values:
\begin{align}
    p_\text{A} &= 0.1592 \\
    p_\text{B} &= 0.1353 \\
    p_\text{C} &= 0.1520 \\
    p_\text{D} &= 0.0239 \\
    p_\text{E} &= 0.0477 \\
    p_\text{F} &= 0.0912 \\
    p_\text{G} &= 0.1918 \\
    p_\text{H} &= 0.0199 \\
\end{align}
The order that the states should be tested in is
\begin{equation}
    H \rightarrow G \rightarrow A \rightarrow C \rightarrow B \rightarrow F \rightarrow E \rightarrow D
\end{equation}

\section{Q2}
For unknown relationships:
\begin{align}
    \text{\# of test cases} &= |A| \times |B| \times |C| \times |D| \times |E| \times |F| \\
    &= 5 \times 5 \times 3 \times 5 \times 2 \times 3 \\
    &= 2250 \\
\end{align}
Knowing the following functional relationships,
\begin{align}
    x &=  A \times D \times E \\
      &= 5 \times 5 \times 2 \\
      &= 50 \\\\
    y &= B \\
    &= 5 \\\\
    z &= C \times F \\
    &= 3 \times 3 \\
    &= 9\\
\end{align}
the upper bound of the number of test cases is:
\begin{align}
    \text{\# of test cases} &= x + y + z \\
    &= 50 + 5 + 9 \\
    &= 64 \\
\end{align}
The following tables are an example set of test cases. Note that every test case requires a value for A, B, C, D, E, F. If a value is not provided for a variable in the proposed test case below, it means that any of the possible values may be selected.
\begin{table}[H]
    \centering
    \caption{Test cases for x}
    \begin{tabular}{|r|r|r|l|r|r|r|l|r|r|r|l|}
    \hline
    \multicolumn{1}{|l|}{Test \#} & \multicolumn{1}{l|}{A} & \multicolumn{1}{l|}{D} & E & \multicolumn{1}{l|}{Test \#} & \multicolumn{1}{l|}{A} & \multicolumn{1}{l|}{D} & E & \multicolumn{1}{l|}{Test \#} & \multicolumn{1}{l|}{A} & \multicolumn{1}{l|}{D} & E \\ \hline
    1                             & 0                      & 7                      & Y & 18                           & 1                      & 10                     & N & 35                           & 3                      & 9                      & Y \\ \hline
    2                             & 0                      & 7                      & N & 19                           & 1                      & 11                     & Y & 36                           & 3                      & 9                      & N \\ \hline
    3                             & 0                      & 8                      & Y & 20                           & 1                      & 11                     & N & 37                           & 3                      & 10                     & Y \\ \hline
    4                             & 0                      & 8                      & N & 21                           & 2                      & 7                      & Y & 38                           & 3                      & 10                     & N \\ \hline
    5                             & 0                      & 9                      & Y & 22                           & 2                      & 7                      & N & 39                           & 3                      & 11                     & Y \\ \hline
    6                             & 0                      & 9                      & N & 23                           & 2                      & 8                      & Y & 40                           & 3                      & 11                     & N \\ \hline
    7                             & 0                      & 10                     & Y & 24                           & 2                      & 8                      & N & 41                           & 4                      & 7                      & Y \\ \hline
    8                             & 0                      & 10                     & N & 25                           & 2                      & 9                      & Y & 42                           & 4                      & 7                      & N \\ \hline
    9                             & 0                      & 11                     & Y & 26                           & 2                      & 9                      & N & 43                           & 4                      & 8                      & Y \\ \hline
    10                            & 0                      & 11                     & N & 27                           & 2                      & 10                     & Y & 44                           & 4                      & 8                      & N \\ \hline
    11                            & 1                      & 7                      & Y & 28                           & 2                      & 10                     & N & 45                           & 4                      & 9                      & Y \\ \hline
    12                            & 1                      & 7                      & N & 29                           & 2                      & 11                     & Y & 46                           & 4                      & 9                      & N \\ \hline
    13                            & 1                      & 8                      & Y & 30                           & 2                      & 11                     & N & 47                           & 4                      & 10                     & Y \\ \hline
    14                            & 1                      & 8                      & N & 31                           & 3                      & 7                      & Y & 48                           & 4                      & 10                     & N \\ \hline
    15                            & 1                      & 9                      & Y & 32                           & 3                      & 7                      & N & 49                           & 4                      & 11                     & Y \\ \hline
    16                            & 1                      & 9                      & N & 33                           & 3                      & 8                      & Y & 50                           & 4                      & 11                     & N \\ \hline
    17                            & 1                      & 10                     & Y & 34                           & 3                      & 8                      & N & \multicolumn{1}{l|}{}        & \multicolumn{1}{l|}{}  & \multicolumn{1}{l|}{}  &   \\ \hline
    \end{tabular}
    \end{table}
    \begin{table}[H]
        \centering
        \caption{Test cases for y}
        \begin{tabular}{|r|l|}
        \hline
        \multicolumn{1}{|l|}{Test \#} & B \\ \hline
        51                            & A \\ \hline
        52                            & B \\ \hline
        53                            & C \\ \hline
        54                            & D \\ \hline
        55                            & E \\ \hline
        \end{tabular}
    \end{table}
    
    \begin{table}[H]
        \centering
        \caption{Test cases for z}
        \begin{tabular}{|r|r|l|}
        \hline
        \multicolumn{1}{|l|}{Test \#} & \multicolumn{1}{l|}{C} & F                     \\ \hline
        56                            & 100                    & $\alpha$ \\ \hline
        57                            & 100                    & $\beta$  \\ \hline
        58                            & 100                    & $\gamma$ \\ \hline
        59                            & 200                    & $\alpha$ \\ \hline
        60                            & 200                    & $\beta$  \\ \hline
        61                            & 200                    & $\gamma$ \\ \hline
        62                            & 300                    & $\alpha$ \\ \hline
        63                            & 300                    & $\beta$  \\ \hline
        64                            & 300                    & $\gamma$ \\ \hline
        \end{tabular}
        \end{table}
\section{Q3}
For simplicity, we will simplify the following keywords:
\begin{itemize}
    \item HARDKEYBOARDHIDDEN = HKH
    \item KEYBOARDHIDDEN = KH 
    \item KEYBOARD = K
    \item NAVIGATIONHIDDEN = NH 
    \item NAVIGATION = N
    \item ORIENTATION = O
    \item SCREENLAYOUT\_LONG = SLL
    \item SCREENLAYOUT\_SIZE = SLS
    \item TOUCHSCREEN = T
    \item UNDEFINED = N/A
    \item LANDSCAPE = LSP 
    \item QWERTY = Q
    \item NOTOUCH = NT 
    \item PORTRAIT = P 
    \item TRACKBALL = TB
\end{itemize}

\scriptsize
\begin{tabularx}{\textwidth}{|X|X|X|X|X|X|X|X|X|X|}
    \caption{Results using pairwise testing} \\ \hline
    \textbf{\#} & \textbf{HKH} & \textbf{KH} & \textbf{K} & \textbf{NH} & \textbf{N} & \textbf{O} & \textbf{SLL} & \textbf{SLS} & \textbf{T} \\ \hline
    1 & NO & NO & 12KEY & NO & DPAD & LSP & MASK & LRG & FNGR \\ \hline
    2 & N/A & N/A & NOKEYS & N/A & NONAV & P & NO & MASK & FNGR \\ \hline
    3 & YES & YES & Q & YES & TB & SQR & N/A & NORM & FNGR \\ \hline
    4 & YES & N/A & N/A & NO & N/A & N/A & YES & SML & NT \\ \hline
    5 & N/A & NO & N/A & YES & WHEEL & N/A & N/A & N/A & STYLUS \\ \hline
    6 & NO & YES & NOKEYS & N/A & WHEEL & SQR & YES & N/A & N/A \\ \hline
    7 & NO & N/A & Q & YES & N/A & LSP & NO & LRG & N/A \\ \hline
    8 & N/A & YES & 12KEY & NO & NONAV & SQR & MASK & SML & STYLUS \\ \hline
    9 & YES & NO & NOKEYS & N/A & TB & LSP & MASK & MASK & NT \\ \hline
    10 & NO & NO & Q & NO & NONAV & P & YES & NORM & NT \\ \hline
    11 & YES & YES & 12KEY & YES & DPAD & P & NO & N/A & NT \\ \hline
    12 & N/A & N/A & 12KEY & N/A & DPAD & N/A & N/A & NORM & N/A \\ \hline
    13 & N/A & N/A & Q & N/A & WHEEL & LSP & MASK & SML & STYLUS \\ \hline
    14 & NO & YES & N/A & N/A & N/A & P & N/A & LRG & STYLUS \\ \hline
    15 & N/A & NO & N/A & NO & TB & SQR & NO & SML & N/A \\ \hline
    16 & YES & YES & NOKEYS & YES & NONAV & N/A & N/A & MASK & N/A \\ \hline
    17 & N/A & N/A & NOKEYS & NO & N/A & SQR & YES & LRG & NT \\ \hline
    18 & NO & N/A & N/A & NO & TB & N/A & MASK & N/A & FNGR \\ \hline
    19 & YES & YES & N/A & NO & WHEEL & LSP & N/A & NORM & NT \\ \hline
    20 & YES & NO & 12KEY & YES & N/A & P & MASK & NORM & N/A \\ \hline
    21 & YES & YES & 12KEY & YES & WHEEL & P & YES & MASK & STYLUS \\ \hline
    22 & NO & YES & NOKEYS & NO & DPAD & SQR & YES & MASK & STYLUS \\ \hline
    23 & YES & YES & Q & YES & DPAD & N/A & NO & SML & STYLUS \\ \hline
    24 & YES & YES & N/A & NO & NONAV & LSP & YES & N/A & FNGR \\ \hline
    25 & YES & YES & N/A & NO & DPAD & N/A & N/A & LRG & STYLUS \\ \hline
    26 & NO & YES & NOKEYS & NO & TB & P & YES & SML & STYLUS \\ \hline
    27 & YES & YES & Q & NO & WHEEL & P & NO & NORM & STYLUS \\ \hline
    28 & YES & YES & Q & NO & N/A & P & N/A & N/A & FNGR \\ \hline
    29 & YES & YES & N/A & NO & WHEEL & P & N/A & SML & FNGR \\ \hline
    30 & YES & YES & N/A & NO & WHEEL & P & N/A & MASK & STYLUS \\ \hline
    31 & YES & YES & Q & NO & WHEEL & P & N/A & MASK & STYLUS \\ \hline
    32 & YES & YES & NOKEYS & NO & WHEEL & P & N/A & NORM & STYLUS \\ \hline
    33 & YES & YES & 12KEY & NO & TB & P & N/A & LRG & STYLUS \\ \hline
    34 & YES & YES & N/A & NO & WHEEL & P & N/A & LRG & STYLUS \\ \hline
    35 & YES & YES & N/A & NO & N/A & P & N/A & MASK & STYLUS \\ \hline
    36 & YES & YES & N/A & NO & NONAV & P & N/A & LRG & STYLUS \\ \hline
\end{tabularx}
\normalsize
There are 36 tests using pairwise testing. This is significantly less than the 172800 tests when considering all combinations. \\\\
To calculate the number of tests required to consider all combinations of input values, we need to perform the following calculation:
\begin{align}
    \text{\# of tests} &= 3 \times 3 \times 4 \times 3 \times 5 \times 4 \times 4 \times 5 \times 4\\
    &= 172800 \\
\end{align}
\section{Q4}
For terminal symbol coverage:
\begin{align}
    \text{\# of tests} &= \text{\# of ops} + \text{\# of letters} + \text{\# of digits} \\
    &= 4 + 52 + 10 \\
    &= 66 \\
\end{align}
For production coverage:
\begin{align}
    \text{\# of tests} &= \text{\# of expr possibilities} + \text{\# of id possibilities} \\ 
    &+ \text{\# of num possibilities} + \text{\# of symbol possibilities}\\
    &= 3 + 2 + 2 + 66 \\
    &= 73 \\
\end{align}

For derivation coverage, there are an infinite amount of tests because of the first expr rule

\begin{table}[H]
    \centering
    \caption{Possible test cases}
    \begin{tabular}{|r|l|}
    \hline
    \multicolumn{1}{|l|}{Test \#} & Test case               \\ \hline
    1                             & a                       \\ \hline
    3                             & ab                      \\ \hline
    4                             & 6  \\ \hline
    5                             & 67 \\ \hline
    6                             & A+4                     \\ \hline
    7                             & AB + 47                 \\ \hline    8                             & A + 47                 \\ \hline
    7                             & B + 7                 \\ \hline
    \end{tabular}
    \end{table}
\end{document}  