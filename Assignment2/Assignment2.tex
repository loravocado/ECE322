\documentclass[12pt, letterpaper, titlepage]{article}
\usepackage[utf8]{inputenc}
\usepackage{geometry}
\usepackage{color,graphicx,overpic} 
\usepackage{fancyhdr}
\usepackage{amsmath,amsthm,amsfonts,amssymb}
\usepackage{mathtools}
\usepackage{hyperref}
\usepackage{multicol}
\usepackage{array}
\usepackage{float}
\usepackage{blindtext}
\usepackage{longtable}
\usepackage{scrextend}
\usepackage[font=small,labelfont=bf]{caption}
\usepackage[framemethod=tikz]{mdframed}
\usepackage{calc}
\usepackage{titlesec}
\usepackage{listings}
\usepackage[normalem]{ulem}
\usepackage{tabularx}
\usepackage{mathrsfs}
\usepackage{bookmark}
\usepackage{setspace}

\mathtoolsset{showonlyrefs}  
\allowdisplaybreaks

\definecolor{mycolor}{rgb}{0, 0, 0}

\geometry{top=2.54cm, left=2.54cm, right=2.54cm, bottom=2.54cm}
\setlength{\headheight}{20pt}
\setlength{\parskip}{0.3cm}
\setlength{\parindent}{1cm}

\pagestyle{fancy}
\fancyhf{}
\rhead{Lora Ma - 1570935}
\lhead{\textit{ECE 322 Assignment 2}}
\rfoot{Page \thepage}

\begin{document} 
\singlespacing

\section{Q1}
Coincidental correctness occurs when no failure is detected even though a fault has been executed. For example, if we wanted to implement $cos(yx)$ and we wanted to test $y = 2$ and $x = 0$, we would expect the program to calculate the answer through something like $cos((2)(0)) = cos(0) = 1$. Had our implementation been incorrect and be something like $cos(\frac{1}{2}yx)$, then the output would still be 0. So although in this case, it returned the correct answer, the implementation is incorrect. This is just a coincidence that the answer is right for this scenario.

\section{Q2}
\begin{enumerate}
    \item Maximal number of rules = 12. This is calculated using: Gender(2 options) * City dwelling(2 options) * Age groups(3 options) = 2*2*3 = 12 \\
    \begin{tabular}{|l|l|l|l|l|l|l|l|l|l|l|l|l|}
    \hline
    Rule \#                                              & 1             & 2             & 3             & 4             & 5                                                                             & 6                                                                            & 7                                                                            & 8                                                                            & 9                & 10               & 11               & 12               \\ \hline
    Gender                                               & M             & F             & M             & F             & M                                                                             & F                                                                            & M                                                                            & F                                                                            & M                & F                & M                & F                \\ \hline
    City                                                 & Yes           & Yes           & No            & No            & Yes                                                                           & Yes                                                                          & No                                                                           & No                                                                           & Yes              & Yes              & No               & No               \\ \hline
    \begin{tabular}[c]{@{}l@{}}Age \\ Group\end{tabular} & \textless{}25 & \textless{}25 & \textless{}25 & \textless{}25 & \begin{tabular}[c]{@{}l@{}}\textless{}65 \&\\ \textgreater{}= 25\end{tabular} & \begin{tabular}[c]{@{}l@{}}\textless{}65 \&\\ \textgreater{}=25\end{tabular} & \begin{tabular}[c]{@{}l@{}}\textless{}65 \&\\ \textgreater{}=25\end{tabular} & \begin{tabular}[c]{@{}l@{}}\textless{}65 \&\\ \textgreater{}=25\end{tabular} & \textgreater{}65 & \textgreater{}65 & \textgreater{}65 & \textgreater{}65 \\ \hline
    A                                                    & X             &               &               &               & X                                                                             &                                                                              &                                                                              &                                                                              & X                &                  &                  &                  \\ \hline
    B                                                    & X             &               & X             &               &                                                                               &                                                                              &                                                                              &                                                                              &                  &                  &                  &                  \\ \hline
    C                                                    &               &               &               &               &                                                                               &                                                                              &                                                                              & X                                                                            &                  &                  &                  &                  \\ \hline
    D                                                    & X             & X             & X             & X             & X                                                                             & X                                                                            & X                                                                            & X                                                                            &                  & X                &                  & X                \\ \hline
    \end{tabular}
    \item Simplified table \\
    \begin{tabular}{|l|l|l|l|l|l|l|l|l|l|l|}
        \hline
        Rule \#                                             & 1             & 2 & 3             & 4             & 5                                                                           & 6                                                                           & 7                                                                           & 8                & 9                & 10               \\ \hline
        Gender                                              & M             & F & M             & F             & M                                                                           & M                                                                           & F                                                                           & M                & F                & M                \\ \hline
        City                                                & Y             & Y & N             & N             & Y                                                                           & N                                                                           & N                                                                           & Y                & Y                & N                \\ \hline
        \begin{tabular}[c]{@{}l@{}}Age\\ Group\end{tabular} & \textless{}25 & - & \textless{}25 & \textless{}25 & \begin{tabular}[c]{@{}l@{}}\textless{}65 \&\\ \textgreater{}25\end{tabular} & \begin{tabular}[c]{@{}l@{}}\textless{}65 \&\\ \textgreater{}25\end{tabular} & \begin{tabular}[c]{@{}l@{}}\textless{}65 \&\\ \textgreater{}25\end{tabular} & \textgreater{}65 & \textgreater{}65 & \textgreater{}65 \\ \hline
        A                                                   & X             &   &               &               & X                                                                           &                                                                             &                                                                             & X                &                  &                  \\ \hline
        B                                                   & X             &   & X             &               &                                                                             &                                                                             &                                                                             &                  &                  &                  \\ \hline
        C                                                   &               &   &               &               &                                                                             &                                                                             & X                                                                           &                  &                  &                  \\ \hline
        D                                                   & X             & X & X             & X             & X                                                                           & X                                                                           & X                                                                           &                  & X                &                  \\ \hline
        \end{tabular}
\end{enumerate}
\section{Q3}
The subdomain is bounded by the linear functions
\begin{align}
    y \geq -x \\
    y < 6 \\ 
    x-2 \leq y \\
    x  > 1 \\
    z > 0 \\
    z < 6 \\
\end{align}
\subsection*{EPC Strategy}
The EPC strategy places the test points at the corners of the shape. The min for x is 1 and the max for x is 8. The min for y is -1 and the max for y is 6. The min for z is 0 and the max is 6. We are expecting $4^3 + 1 = 65$ test cases. The extreme points chosen are (8, 8.1, 1, 0.9) for x, (6, 6.1, -1, -1.1) for y, and (6, 6.1, 0, -0.1) \\
    \begin{tabular}{|l|l|l|l|l|l|l|l|l|l|l|l|l|l|l|}
    \hline
    Test & x   & y    & z    & Pass? & Test & x   & y    & z    & Pass? & Test & x   & y    & z    & Pass? \\ \hline
    1    & 8   & 6    & 6    & No       & 23   & 8.1 & 6.1  & -1   & No    & 45   & 1   & -1.1 & 6    & No    \\ \hline
    2    & 8   & 6    & 6.1  & No       & 24   & 8.1 & 6.1  & -1.1 & No    & 46   & 1   & -1.1 & 6.1  & No    \\ \hline
    3    & 8   & 6    & -1   & No       & 25   & 8.1 & -1   & 6    & No    & 47   & 1   & -1.1 & -1   & No    \\ \hline
    4    & 8   & 6    & -1.1 & No       & 26   & 8.1 & -1   & 6.1  & No    & 48   & 1   & -1.1 & -1.1 & No    \\ \hline
    5    & 8   & 6.1  & 6    & No       & 27   & 8.1 & -1   & -1   & No    & 49   & 0.9 & 6    & 6    & No    \\ \hline
    6    & 8   & 6.1  & 6.1  & No       & 28   & 8.1 & -1   & -1.1 & No    & 50   & 0.9 & 6    & 6.1  & No    \\ \hline
    7    & 8   & 6.1  & -1   & No       & 29   & 8.1 & -1.1 & 6    & No    & 51   & 0.9 & 6    & -1   & No    \\ \hline
    8    & 8   & 6.1  & -1.1 & No       & 30   & 8.1 & -1.1 & 6.1  & No    & 52   & 0.9 & 6    & -1.1 & No    \\ \hline
    9    & 8   & -1   & 6    & No       & 31   & 8.1 & -1.1 & -1   & No    & 53   & 0.9 & 6.1  & 6    & No    \\ \hline
    10   & 8   & -1   & 6.1  & No       & 32   & 8.1 & -1.1 & -1.1 & No    & 54   & 0.9 & 6.1  & 6.1  & No    \\ \hline
    11   & 8   & -1   & -1   & No       & 33   & 1   & 6    & 6    & No    & 55   & 0.9 & 6.1  & -1   & No    \\ \hline
    12   & 8   & -1   & -1.1 & No       & 34   & 1   & 6    & 6.1  & No    & 56   & 0.9 & 6.1  & -1.1 & No    \\ \hline
    13   & 8   & -1.1 & 6    & No       & 35   & 1   & 6    & -1   & No    & 57   & 0.9 & -1   & 6    & No    \\ \hline
    14   & 8   & -1.1 & 6.1  & No       & 36   & 1   & 6    & -1.1 & No    & 58   & 0.9 & -1   & 6.1  & No    \\ \hline
    15   & 8   & -1.1 & -1   & No       & 37   & 1   & 6.1  & 6    & No    & 59   & 0.9 & -1   & -1   & No    \\ \hline
    16   & 8   & -1.1 & -1.1 & No       & 38   & 1   & 6.1  & 6.1  & No    & 60   & 0.9 & -1   & -1.1 & No    \\ \hline
    17   & 8.1 & 6    & 6    & No       & 39   & 1   & 6.1  & -1   & No    & 61   & 0.9 & -1.1 & 6    & No    \\ \hline
    18   & 8.1 & 6    & 6.1  & No       & 40   & 1   & 6.1  & -1.1 & No    & 62   & 0.9 & -1.1 & 6.1  & No    \\ \hline
    19   & 8.1 & 6    & -1   & No       & 41   & 1   & -1   & 6    & No    & 63   & 0.9 & -1.1 & -1   & No    \\ \hline
    20   & 8.1 & 6    & -1.1 & No       & 42   & 1   & -1   & 6.1  & No    & 64   & 0.9 & -1.1 & -1.1 & No    \\ \hline
    21   & 8.1 & 6.1  & 6    & No       & 43   & 1   & -1   & -1   & No    & 65   & 1   & 1    & 1    & No    \\ \hline
    22   & 8.1 & 6.1  & 6.1  & No       & 44   & 1   & -1   & -1.1 & No    &      &     &      &      &       \\ \hline
    \end{tabular}
    \subsection*{Weak Nx1 Strategy}
    For the weak nx1 strategy \\
        \begin{tabular}{|r|r|r|r|l|}
        \hline
        \multicolumn{1}{|l|}{Test Number} & \multicolumn{1}{l|}{x} & \multicolumn{1}{l|}{y} & \multicolumn{1}{l|}{z} & Passed? \\ \hline
        1                                 & 1                      & 0                      & 1                      & No      \\ \hline
        2                                 & 6                      & 3                      & 4                      & No      \\ \hline
        3                                 & 6                      & 5                      & 3                      & Yes     \\ \hline
        4                                 & 6                      & 6                      & 2                      & No      \\ \hline
        5                                 & 2                      & 6                      & 5                      & No      \\ \hline
        6                                 & 7                      & 5                      & 0                      & No      \\ \hline
        7                                 & 2                      & 3                      & 3                      & Yes     \\ \hline
        8                                 & 7                      & 5                      & 6                      & No      \\ \hline
        9                                 & 3                      & 2                      & 5                      & Yes     \\ \hline
        10                                & 1                      & 4                      & 4                      & No      \\ \hline
        11                                & 4                      & 4                      & 0                      & No      \\ \hline
        12                                & 4                      & 6                      & 4                      & No      \\ \hline
        13                                & 1                      & 6                      & 5                      & No      \\ \hline
        14                                & 4                      & 4                      & 6                      & No      \\ \hline
        15                                & 3                      & 2                      & 1                      & Yes     \\ \hline
        16                                & 3                      & 1                      & 6                      & No      \\ \hline
        17                                & 3                      & 1                      & 2                      & Yes     \\ \hline
        18                                & 3                      & 1                      & 0                      & No      \\ \hline
        19                                & 5                      & 3                      & 3                      & Yes     \\ \hline
        20                                & 1                      & -1                     & 1                      & Yes     \\ \hline
        21                                & 2                      & 3                      & 0                      & No      \\ \hline
        22                                & 7                      & 5                      & 5                      & Yes     \\ \hline
        23                                & 3                      & 2                      & 3                      & Yes     \\ \hline
        24                                & 2                      & 3                      & 6                      & No      \\ \hline
        25                                & 8                      & 6                      & 1                      & No      \\ \hline
        26                                & 1                      & 2                      & 2                      & No      \\ \hline
        \end{tabular}


For the next subdomain, there are 4 dimensions ($x, y, z, w$) making us require a total of $4^4 + 1 = 257$ tests

\section{Q4}
In a quadratic equation, there are always two roots. In general, to solve for these roots, you can use the quadratic equation where the roots will always be one of the three: real and distinct, real and equal, or complex. The discriminant is the expression $b^2 - 4ac$ for any quadratic equation $ax^2+bx+c =0$. Based on the sign of this expression, you can determine how many real number solutions the quadratic equation has. This expression is not a linear boundary and is not ideal when weak nx1 testing. To use weak nx1 testing in this case, we would need to have many subdivisions. Further, EPC testing is also not ideal since the domain for a quadratic function is always $(-\infty, \infty)$. This makes it impossible to choose a min and max value. If we bound the inputs, then EPC testing would be possible.
\section{Q5}
\begin{align}
    P(f_1) &= (0. 5)(0. 3)(0. 0) + (0. 5)(0. 7)(0. 1) + (0. 1)(0. 6)(0. 5) + (0. 1)(0. 4)(0. 1) \\&+ (0. 4)(0. 7)(0. 1) + (0.4)(0.3)(0.0) \\
           &= 0 + 0. 035 + 0. 03 + 0. 004 + 0. 028 + 0 \\
           &= 0.097
\end{align}
\begin{align}
    P(f_2) &=  (0. 5)(0. 3)(0. 3) + (0. 5)(0. 7)(0. 1) + (0. 1)(0. 6)(0. 1) + (0. 1)(0. 4)(0. 9)  \\&+ (0. 4)(0. 7)(0. 3) + (0.4)(0.3)(0.8) \\
           &=  0. 045 + 0. 035 + 0. 006 + 0. 036 + 0. 084 + 0. 096 \\
           &= 0. 302
\end{align}
\begin{align}
    P(f_3) &= (0. 5)(0. 3)(0. 5) + (0. 5)(0. 7)(0. 3) + (0. 1)(0. 6)(0. 0) + (0. 1)(0. 4)(0. 0)  \\&+  (0. 4)(0. 7)(0. 2) + (0. 4)(0.3)(0.0) \\
           &=  0. 075 + 0. 105 + 0 + 0 + 0. 056 + 0 \\
           &= 0.236
\end{align}
\begin{align}
    P(f_4) &= (0. 5)(0. 3)(0. 2) + (0. 5)(0. 7)(0. 5) + (0. 1)(0. 6)(0. 4) + (0. 1)(0. 4)(0. 0)  \\&+  (0. 4)(0. 7)(0. 4) + (0. 4)(0.3)(0.2) \\
           &=  0. 03 + 0. 175 + 0. 024 + 0 + 0. 112 + 0. 024 \\
           &= 0.365
\end{align}

Therefore $f_4 > f_2 > f_3 > f_1$

\end{document}